\documentclass{article}
\usepackage{amsmath}

\title{What is a Model?}
\author{}
\date{}

\begin{document}

\maketitle

\section{Types of Models}

\subsection{Physical Lab Model}
\textbf{Description:} A tangible, scaled representation of a real-world system used for experimental testing. Examples include flume experiments for river hydraulics and wind tunnel tests for aerodynamics.

\subsection{Interpretive Model}
\textbf{Description:} A framework used to explain or assign meaning to observed phenomena. Example: A geological interpretation of past climate conditions from sediment cores.

\subsection{Perceptual Model}
\textbf{Description:} A mental model or intuitive framework that influences how scientists or practitioners interpret data and processes.



\subsection{Statistical Model}
\textbf{Description:} A probabilistic framework describing relationships between variables, incorporating uncertainty.

\[
P(Y | X) = \frac{P(X | Y) P(Y)}{P(X)}
\]

Kriging, for example:


\subsection{Empirical Model}
\textbf{Description:} A model based purely on observed data rather than first principles. Often expressed as a fitted function.

\[
y = ax + b
\]

variogram models, for example


\subsection{Machine Learning Model}
\textbf{Description:} A flexible data-driven model that learns patterns from data, often using neural networks or decision trees.

\[
\hat{y} = f(X; \theta)
\]

\subsection{Conceptual Model}
\textbf{Description:} A simplified, qualitative representation of a system, often expressed as flowcharts or box-and-arrow diagrams. Example: The water cycle model.

\textbf{Linear Reservoir Model:} A simple conceptual hydrologic model where the outflow \( Q \) is proportional to the storage \( S \):

\[
\frac{dS}{dt} = I - Q
\]

where:
- \( S \) is the storage in the reservoir (e.g., groundwater, lake, or soil moisture).
- \( I \) is the inflow (e.g., precipitation, recharge).
- \( Q \) is the outflow, assumed proportional to storage: 

\[
Q = \frac{S}{K}
\]

where \( K \) is the reservoir **response time** (a storage coefficient with units of time).

\textbf{Discrete Solution:} Using **forward Euler approximation**:

\[
S^{n+1} = S^n + \Delta t (I^n - Q^n)
\]

Substituting \( Q^n = S^n / K \):

\[
S^{n+1} = S^n + \Delta t \left( I^n - \frac{S^n}{K} \right)
\]

\textbf{Outflow Update:}

\[
Q^{n+1} = \frac{S^{n+1}}{K}
\]

where:
- \( S^n \) is storage at time step \( n \).
- \( Q^n \) is outflow at time step \( n \).
- \( \Delta t \) is the time step.

This equation provides a simple way to track how water moves through a storage unit (e.g., watershed, aquifer) using time-stepping.

\subsection{Physical Model (PDE-Based)}
\textbf{Description:} A computational simulation using physical laws, often represented as partial differential equations (PDEs). Used in climate models, fluid dynamics, and geophysics.

\[
\frac{\partial u}{\partial t} = \alpha \frac{\partial^2 u}{\partial x^2}
\]

where:
- \( u(x,t) \) is the temperature at position \( x \) and time \( t \).
- \( \alpha \) is the thermal diffusivity.

\subsection{Physical Computer Model (PDE-Based)}
\textbf{Description:} A computational simulation using physical laws, often represented as partial differential equations (PDEs). Used in climate models, fluid dynamics, and geophysics.

\[
\frac{u_i^{n+1} - u_i^n}{\Delta t} = \alpha \frac{u_{i+1}^n - 2u_i^n + u_{i-1}^n}{\Delta x^2}
\]

\textbf{Solution Update:} Solving for \( u_i^{n+1} \):

\[
u_i^{n+1} = u_i^n + \frac{\alpha \Delta t}{\Delta x^2} \left( u_{i+1}^n - 2u_i^n + u_{i-1}^n \right)
\]

where:
- \( u_i^n \) is the temperature at grid point \( i \) and time step \( n \).
- \( \Delta t \) is the time step size.
- \( \Delta x \) is the spatial step size.



\section{What Are Models Used For?}

\subsection{Reconstructing the Past (Hindcasts)}
\textbf{Example:} Using tree rings to infer past climate conditions.

\subsection{Understanding the Present (Nowcasts)}
\textbf{Example:} Real-time flood models predicting river discharge.

\subsection{Predicting the Future (Forecasts)}
\textbf{Example:} Weather forecasting using atmospheric models.

\subsection{Interpolating (Filling in the Gaps)}
\textbf{Example:} Using kriging to estimate missing spatial data.

\subsection{Extrapolating to Unknown Events}
\textbf{Example:} Projecting sea level rise based on ice sheet dynamics.

\subsection{Understanding the Physical World}
\textbf{Example:} Modeling plate tectonics to understand earthquake hazards.

\subsection{Analyzing Counterfactuals}
\textbf{Example:} Simulating the impact of removing a dam on river flow.

\subsection{Attributing Influence from Variables, Events, or Forcings}
\textbf{Example:} Determining the role of CO$_2$ in historical climate change.

\subsection{Identifying Causal Relationships}
\textbf{Example:} Using statistical inference to assess how deforestation affects local precipitation.

\end{document}